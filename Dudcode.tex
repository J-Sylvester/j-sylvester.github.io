<h3>1. Provide clear detailed slides, preferably also lecture notes </h3>

<table border="0"><tr><td width="70%">
<tr>
<td>
<p> This first point is the most obvious but arguably the most important. A learning outcome of a theory course is that the students should understand and analyse algorithms and models, this is difficult if the algorithm/model is not clearly described in the course material. Slides are the primary method of content delivery in Computer Science, this has advantages and disadvantages. The main disadvantage is that long proofs have to either be summarised/simplified or broken over several slides. The former degrades the quality of slides as a learning resource, which is not ideal if they are the primary resource. The second option has serious impacts for engagement during lectures. The current set of (inherited) slides on my course is optimised for lecturing, and not for use as a primary text. The idea is that the students should refer to one of the course textbooks for a more formal treatment of the material, however the average student will not refer to a textbook, and will just rely on the slides.
</p>

<p>  A potential solution to the issues present in slides is to also produce lecture notes. I am currently working on revamping the slides (converting them from ppt to beamer) and although my inital thought was to make them more complete/detailed I am moving towards keeping them roughly as they are and producing a set of lecture notes alongside them. My hope is that as these notes contain only what is in the course (just in more detail) and will be written by me, then the students will see they are directly relevant to them passing the course, and will find them more accessable than a larger and more general textbook.
</p>

</td>
<td>  <div class="sideimg">
<img  src="/home/john/Documents/PGCAP/Patch 3/scroll.webp" alt="Scroll">
</div>
</td>
</tr>
</table>
<br>